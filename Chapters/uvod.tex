% Na 1 A4. Tady se rozepíše zadání, "dá se mu tvar", zasadí se do obecného
% rámce. Tedy, ne že bych úplně rozuměl těm dvěma posledním pojmům.

V posledních letech se zvyšuje zájem o sběr dat ze senzorů nasazených v odlehlých lokalitách, 
kde jedinou možností napájení jsou zabudované baterie (popř. různé formy energy harvestingu),
a jediným komunikačním kanálem je rádiové spojení.

Na tyto požadavky odpovídají sítě souhrnně nazývané LPWAN, tj. nízkoenergetické sítě velké rozlohy.
Umožňují konstrukci zařízení která posílají data na vzdálenosti v řádu až desítek kilometrů, při výdrži na zabudovanou baterii v řádech let, popř. s možností napájení alternativními zdroji energie --- solární články, termoelektrické zdroje atp.

Jako zajímavé se jeví použití těchto sítí při aktivitách volnočasových organizací pracujícími
s dětmi a mládeží. Zařízení schopné odesílat digitální, strukturovaná data na delší vzdálenosti
mohou zastat úkoly které, ač jsou jednoduché, vyžadují fyzickou přítomnost vedoucího.
Typickým příkladem je kontrola zda hráči došli na dané místo v daný čas a splnili zadaný úkol.
Tato práce se zabývá konstrukcí takového zařízení.

Pro tento účel je, netypicky, "nízkopříkonovost" zařízení zcela nevýznamná. Předpokládá se že zařízení bude instalováno na jedno místo na dobu maximálně několika týdnů (letní tábory), a že může být v budoucnu rozšířeno o funkce které jsou energeticky náročné ze své podstaty --- např. reprodukce zvuků, osvětlování okolí, pohyb předměty pomocí motorů a serv atp.

LPWAN komunikace je zvolena pro její další výhody --- umožňuje pokrýt potřebné oblasti
signálem svépomocí, příslušné rádiové moduly jsou poměrně levné a fungují v bezlicenčních pásmech.

Uživatelské rozhraní je realizováno aplikací pro OS Android, která se zařízením komunikuje skrze NFC.
Díky tomu lze uživatelské rozhraní snadno upravovat, a zároveň se usnadňuje zajištění krytí proti
vlivům prostředí, zejména dešti.


\textbf{TODO: tvrdím zde spoustu věcí --- dohledat citace}


% Běžně dostupné technologie rádiového přenosu pak lze zjednodušeně rozdělit do 
% čtyřech skupin podle spotřeby energie a dosahu (třetím významným parametrem Je
% pak rychlost přenosu).
% Na krátké vzdálenosti v řádu max. desítek metrů lze využít např. WiFi pro 
% vysoké či Bluetooth pro nízké přenosové rychlosti, na dlouhé vzdálenosti v řádu
% kilometrů či desítek kilometrů existují


%     \subsection{Povaha vyvíjeného zařízení}

%     Cílem této práce je návrh a výroba zařízení pro použití při venkovních 
%     volnočasových aktivitách, zejména pak na táborech a víkendových akcích
%     pořádaných organizacemi pro děti a mládež.

%     Vyvíjené zařízení má být schopno komunikovat na velké vzdálenosti (v řádu
%     kilometrů) pomocí LoRa komunikace v síti LoRaWAN, a na krátké vzdálenosti
%     pomocí NFC rozhraní.

%     Smyslem tohoto zařízení je nahrazení jednoduchých úkolů běžně prováděných
%     vedoucími, jako je vyčkávání na určitém místě a kontrola příchodu hráčů.

%     \subsection{Funkcionalita zařízení}

%     V této práci implementuji základní funkcionalitu prokazující funkčnost
%     komunikací skrz NFC a LoRa/LoRaWAN. Pro vývoj firmwaru uvažuji
%     následující scénář použití při organizovaných volnočasových aktivitách:

%     \begin{enumerate}
%         \item Zařízení se umístí do terénu
%         \item Hráči dostanou za úkol se k zařízení dostat
%         \item Hráči k zařízení přijdou, spustí svou aplikaci, vepíšou své jméno
%             a dotknou se NFC portu zařízení
%         \item Aplikace načte skrz NFC ze ze zařízení otázku
%         \item Hráč v aplikaci vepíše odpověď na otázku
%         \item Hráč znovu přiloží mobil k zařízení. Během tohoto přiložení 
%             proběhnou následující akce:
%         \begin{itemize}
%             \item Aplikace skrz NFC zapíše do zařízení hráčovo jméno a jeho
%                 odpověď na otázku
%             \item Zařízení zkontroluje správnost zapsané odpovědi;
%             \item Na základě správnosti odpovědi zařízení přepíše obsah tagu
%                 s dalšími instrukcemi pro hráče
%             \item Aplikace opakovaně čte obsah tagu; jakmile pozná změnu obsahu,
%                 vyčte jej a zobrazí hráči
%         \end{itemize}
%         \item Zařízení zašle informaci s hráčovým jménem a jeho odpovědí skrz
%             LoRaWAN na server TTN.
%         \item Aplikace vedoucích opakovaně vyčítá a zobrazuje data ze serveru
%     \end{enumerate}

% \section{Popis jednotlivých částí projektu}

%     V této práci probíhá komunikace mezi mobilní aplikací na telefonech hráčů,
%     vyvíjeným zařízením, a mobilní aplikací vedoucích skrze LoRaWAN síť.
%     Níže je popsána role každé části projektu.

%     \subsection{Mobilní aplikace}

%         Mobilní aplikace je vyvíjena pro OS Android s použitím oficiálně 
%         podporovaného programovacího jazyka Kotlin.

%         V této jedné aplikaci je implementována funkcionalita jak pro vedoucí,
%         tak pro hráče. Toto stačí pro demostraci funkčnosti řešení, avšak při
%         praktickém použití by jistě bylo vhodné tyto funkce oddělit do 
%         zvláštních aplikací, či alespoň funkce určené pro vedoucí např. 
%         uzamknout přístupovým heslem.

%         Požadavky na mobilní aplikaci jsou:
%         \begin{itemize}
%             \item Z hlediska hráčů:
%             \begin{itemize}
%                 \item Načtení jména hráče
%                 \item Čtení a zápis informací z NFC tagů
%                 \item Zobrazení otázky vyčtené ze zařízení hráči
%                 \item Možnost zápisu odpovědi uživatelem
%                 \item Zobrazení pokynu k pokračování vyčteného ze zařízení
%             \end{itemize}
%             \item Z hlediska vedoucích:
%             \begin{itemize}
%                 \item Internetové připojení
%                 \item Pravidelná kontrola dat na serveru
%             \end{itemize}
%         \end{itemize}

%     \subsection{Vyvíjené zařízení}

%         Vyvíjené zařízení je stěžejní částí práce. Je poskládáno z
%         vývojových kitů, jejich seznam je uveden v \ref{itm:seznam_kitu}

%         Požadavky na vyvíjené zařízení jsou:
%         \begin{itemize}
%             \item Zápis informací na \textbf{dynamický NFC tag}%dát link na to
%             % co to je
%             \item Reakce na čtení z dyn. NFC tagu
%             \item Uchování otázek a jiných informací i po přerušení napájení
%             \item Komunikace pomocí LoRa radiového spojení
%             \item Odesílání dat v rámci LoRaWAN sítě 
%             \item Vlastní zdroj energie
%         \end{itemize}
        
%         Zařízení neodesílá ani neuchovává informaci o aktuálním čase. Je to z
%         toho důvodu že TTN server automaticky označí každou přijatou zprávu
%         časovou známkou, což poskytuje dostatečnou přesnost pro účely aktivit
%         pro něž je zařízení použito.
        
%     \subsection{TTN server}

%         Vyvíjené zařízení komunikuje v rámci sítě LoRaWAN; to zahrnuje síťový
%         server (viz \ref{subsubsec:LoRaWAN_architecture}), provozovaný některým
%         z LoRaWAN operátorů.
        
%         V této práci používám síť The Things Network, která zdarma poskytuje
%         brány a síťový server.

%         Požadavky na tento server jsou:
%         \begin{itemize}
%             \item Možnost zaregistrování vyvíjené LoRaWAN aplikace
%             \item Příjem dat z vyvíjeného zařízení
%             \item Uchování přijatých dat po dostatečně dlouhou dobu a jejich
%                 poskytnutí po zaslání příslušného požadavku
%         \end{itemize}

%         TTN server poskytuje zdarma možnost uchování přijatých dat po 7 dní, a
%         po tu dobu je zpřístupňuje skrz HTTP API. To je pro účely tohoto 
%         projektu dostatečné.