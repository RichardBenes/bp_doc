% Popis technických vlastností a použití rozhraní NFC.

% Jaké jsou NFC tagy?
% Jak se s nimi komunikuje? Jaké jsou tam příkazy?
% Něco k fungování mobilu samotného jako NFC tag
% Co je dynamický NFC tag
% Popis toho co použitý NFC tag umí a co neumí

%Vlastně můžu přepsat co jsem si zapsal --- moc víc toho není.

% Info zde je z doporučené knihy
\section{Obecný popis NFC}

    NFC je technologie umožňující bezdrátovou výměnu krátkých zpráv na 
    krátké vzdálenosti, typicky v jednotkách centimetrů.

    NFC vychází z RFID komunikace, konkrétně z normy ISO-14443 popisující
    komunikaci na frekvenci 13,56 MHz.

    Ačkoli jsou specifikace NFC dostupné (po zakoupení), je tato technologie 
    patentem, proto je nutno pro vývoj a výrobů zařízení tuto technologii 
    užívajících zakoupit příslušnou licenci.

    Komunikace probíhá mezi dvěma zařízeními, v jednom ze dvou režimů:

    \begin{description}
        \item [Aktivní] obě komunikující zařízení mají svůj vlastní zdroj
            energie (typicky baterii) --- například komunikace mezi dvěma mobilními telefony
        \item [Pasivní] jedno z komunikujících zařízení má vlastní zdroj energie
            a generuje rádiový signál, druhé je tímto signálem napájeno a
            odpovídá na zaslané požadavky.
    \end{description}

    Při komunikaci v pasivním režimu se zařízení napájené rádiovým polem obvykle
    označuje jako NFC tag. Tento režim je v kontextu používání mobilních 
    telefonů známější.

\section{Typy NFC tagů}

    Existují 4 typy NFC tagů standardizované NFC Fórem, a několik dalších typů
    definovaných jejich výrobci. Standardizované tagy jsou označeny jako 
    typ 1 až typ 4.

    Jednotlivé typy se navzájem liší zpravidla limity pro paměť, přenosovými 
    rychlostmi a podporovanými příkazy, podporou pro čtení či čtení i zápis a 
    dalšími vlastnostmi, např. možnostmi detekce kolizí při komunikaci.

\section{NDEF}

% TODO: Vzhledem k tomu že řeším parsování obsahu NDEF, bylo by vhodné
% zde jeho strukturu více rozepsat.

    Výrazným rozdílem oproti ostatním RFID technologiím je použití 
    standardizovaného datového formátu, --- NFC Data Exchange Format. Díky tomu
    může vývojář aplikace používající NFC přenechat řešení specifik různých
    typů tagů na použité knihovně, a pracovat pouze se zaslanými/přijatými daty,
    jejichž formát je známý.
    
    Základní jednotkou dat je NDEF Message (zpráva), která
    obsahuje jeden nebo více NFC Record ("záznam"). Tyto záznamy se skládají z
    hlavičky a samotných zasílaných dat.

    Hlavička záznamu obsahuje zejména informace o délce a typu dat v záznamu
    obsažených, navíc pak např. informace o tom, zda se jedná o první či o
    poslední záznam ve zprávě. 

    Z informací o typu (či formátu) dat je nejdůležitější pole označené jako 
    Payload Type, které obsahuje textový řetězec popisující formát dat. 
    Formát toho řetězce však závisí na 3 bitové hdnotě TNF 
    (Type Name Format), která může nabývat osmi hodnot:
    
    \begin{description}
        \item [0 --- Empty] Značí nepřítomnost dat, v tomto případě ani pole 
            Payload Type není přítomno
        \item [1 --- Well-known] Data jsou v jednom z formátů definovaných NFC
            fórem, například (v závorce odpovídající řetězec pole Payload Type)
            \begin{itemize}
                \item Prostý text, obsahující navíc informace o kódování a 
                    jazyce ("T")
                \item URI, tj. internetová adresa. Pro snížení množství dat jsou
                    však některé její části zakódovány dle NDEF standardů ("U")
                \item Smart Poster ("Sp")
            \end{itemize}
        \item [2 --- MIME media type] Například "text/json", "image/gif" atp.
        \item [3 --- Absolute URI] URI, odkazující na dokument popisující formát
            přenášených dat
        \item [4 --- External] Uživatelsky definovaný typ, například Android
            Application Record, obsahující informaci o tom jaká aplikace se má
            po načtení tagu spustit
        \item [5 --- Unknown] Pole Payload Type není přítomno
        \item [6 --- Unchanged] Stejný typ jako u předcházejícího záznamu --- 
            tento typ nemůže být u prvního záznamu ve zprávě
        \item [7 --- Reserved] Rezervováno pro budoucí užití
    \end{description}
    
    Samotná zpráva nemá žádnou hlavičku --- skládá se
    pouze ze záznamů zařazených za sebe, první a poslední záznam jsou pak označeny
    příslušnými bity ve svých hlavičkách, díky čemuž komunikující zařízení
    pozná (do jisté míry) celistvost zprávy.
    První záznam mívá zpravidla zvláštní význam; 
    například OS Android dle prvního záznamu rozhoduje, která z aplikací
    bude spuštěna pro zpracování dat z tagu.


\section{Dynamické NFC tagy}

    Dynamické NFC tagy se od běžných liší možností přistupovat do paměti i 
    jinými způsoby než jen NFC komunikací, a případně i obsah paměti měnit;
    např. v závislosti na zapsaných datech, externích událostech atp.
    Pro generování těchto dat pak bývají tyto tagy doplněny mikrokontrolérem.

    Dynamický NFC tag je použit v této práci.

    % V této práci je použit dynamický NFC tag M24ST od výrobce ST 
    % Microelectronics. Obsahuje EEPROM paměť velikosti 8192B, ke které je možno
    % kromě NFC rozhraní přistupovat skrze I2C sběrnici. Dále je vybaven 
    % digitálním výstupem, pomocí nějž lze detekovat probíhající bezdrátovou 
    % komunikaci.