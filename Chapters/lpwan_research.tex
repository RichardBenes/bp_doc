
LPWAN sítě jsou určeny především pro získávání dat ze zařízení která mají k dispozici omezené množství energie, a potřebují komunikovat rádiově na velké vzdálenosti, a to i v oblastech kde pokrytí mobilními sítěmi telefonních operátorů zcela chybí.

\textbf{TODO: uvést co je to link budget, a jaký je vztah mezi rychlostí, spotřebou a dosahem}

Jejich typickými znaky jsou:

\begin{itemize}
    \item Nízké nároky na energii a tím dlouhá výdrž na baterie, v některých případech i více než 10 let
    \item Nízká cena zařízení, zvláště pak rádiové části
    \item Provoz v bezlicenčních, a tedy neplacených frekvenčních pásmech
    \item Daleký dosah rádiové  komunikace --- v řádu kilometrů
    \item Odolnost proti rušení
    \item Optimalizace komunikace pro posílání krátkých zpráv
    \item Jednoduché komunikační protokoly s malým množstvím metadat
    \item Nízké nároky na výpočetní výkon připojených zařízení
    \item Častější přenos informací ze zařízení do sítě (uplink) než opačně
    \item Hvězdicová topologie, kdy zařízení zasílají data na nejbližší bránu, která je připojena k internetu
\end{itemize}

Běžné síťové protokoly (IP, TCP, HTTP...) jsou navržené pro přenášení velkého 
množství dat (např. maximální velikost IP paketu je 65535 bajtů), čemuž odpovídá
množství přidaných metadat --- např. u IP paketu je minimální velikost hlavičky
20 bajtů. Další zatížení sítě pak často představuje zpětné potvrzování přijatých
dat (TCP, WiFi).
% https://en.wikipedia.org/wiki/IPv4

Protože se v LPWAN sítích běžně posílají data o délkách v jednotkách bajtů,
nejsou běžné protokoly vhodné --- množství metadat protokoly přidaných by 
bylo až násobně vyšší než množství přenášených dat.

Pro efektivní provoz v podmínkách LPWAN sítí proto bývají použité protokoly
charekteristické:

\begin{itemize}
    \item Malým, nebo dokonce téměř žádným množstvím metadat
    \item Dopřednými metodami kontroly chyb, což značně snižuje potřebu zasílání
        potvrzení přijetí, i když za cenu navýšení množství zaslaných dat
    \item Hvězdicovou topologií, kdy koncová zařízení zasílají svá data na 
        nejbližší bránu, ze které jsou pak doručována běžnými internetovými
        technologiemi na příslušný server
\end{itemize}

V této sekci je popsáno několik LPWAN technologií, z nichž jsou dvě 
proprietální (Sigfox a Ingenu), ostatní jsou otevřenými standardy. LoRa a 
LoRaWAN, v této práci použité, jsou rozebrány podrobněji ve vlastních sekcích. 
LoRaWAN je otevřený standard, který podporuje bezdrátový přenos několika kanály
používajícími proprietální modulaci LoRa a jedním kanálem používajícím 
standardní modulaci FSK.

\section{Sigfox}

Sigfox je proprietální technologií francouzské společnosti Sigfox S.A. Tato 
stanovuje topologii podobnou mobilním sítím, ve které jednotlivá zařízení 
komunikují se stanicemi připojenými k internetu. Ty pak posílají přijaté zprávy 
na síťový server, který odstraní duplikáty a pošle data cílovému uživateli.

Fyzický přenos dat funguje v sub-GHz bezlicenčním pásmu (v Evropě ISM pásmo 
868MHz).

Pro přenos směrem od zařízení na stanici (uplink) je použita D-BPSK modulace. 
Každá zpráva je přenášena rychlostí 100 nebo 600 bitů za sekundu, v závislosti 
na regionu. Celkové pásmo ve kterém Sigfox pracuje je široké 192kHz, díky 
použité modulaci pak každá zpráva zabírá z tohoto pásma šířku pouze 1Hz na 1bps, 
tedy 100 nebo 600 Hz.

Každé zařízení může vyslat maximálně 140 zpráv za jeden den, množství dat v 
každé zprávě je do 12 bajtů. Při tomto směru komunikace má Sigfox hodnotu link 
budget $-163,3$dB i při nízkém vysílacím výkonu.
% Co je to link budget?
% jak vysokém/nízkém vysílacím výkonu?

Komunikace směrem od stanice k zařízení (downlink) je omezena na max. 4 zprávy 
za den po 8 bajtech, narozdíl od uplink komunikace používá GFSK modulaci.

Zařízení při odesílání používá frekvenci náhodně vybranou z celého pásma --- 
Sigfox pásmo \emph{nerozděluje} na kanály. Díky tomu jsou kladeny jen nízké 
nároky na přesnost použité frekvence, což příznivě ovlivňuje cenu rádiových částí komunikačních
modulů. Díky úzkému použitému pásmu použitému při vysílání a omezenému počtu 
odesílaných zpráv za den je velmi malá pravděpodobnost kolizí. Zařízení 
přistupují k médiu náhodně, nedochází k žádné synchronizaci mezi zařízením a 
stanicí. Pro zajištění spolehlivého přenosu je každá zpráva odeslána třikrát.

Při přenášení se díky optimalizovanému protokolu přenáší jen malé množství 
metadat --- pro přenos 12 bajtů dat je celková délka zprávy pouze 26 bajtů.

Každé koncové zařízení má v uživateli nepřístupné paměti uložen privátní klíč, a 
v přístupné paměti ID. Každá zpráva obsahuje podpis, vygenerovaný pomocí klíče a 
ID, a sekvenční číslo. Díky tomu je komunikace odolná proti podvrženým zprávám.
%\footnote{Video ``Radio access network'' na https://www.sigfox.com/en/what%
%-sigfox/technology}.
%\footnote{Sekce ``Technical summary'' na https://en.wiki%
%pedia.org/wiki/DASH7}). 
%\href{https://www.sigfox.com/en/what-sigfox/technology}
%    {https://www.sigfox.com/en/what-sigfox/technology}

\section{Ingenu}

Ingenu je společnost poskytující LPWAN konektivitu v bezlicenčním pásmu 2.4GHz, s důrazem na dlouholetou podporu nasazených zařízení. Výhodou 
použití pásma 2.4GHz je jeho dostupnost po celém světě --- na rozdíl od 
technologií používajících sub-GHz pásma tedy není potřeba při nákupu vysílačů 
rozlišovat
region, ve kterém bude zařízení provozováno. Použitá topologie je opět podobná mobilním 
sítím, kdy koncové zařízení komunikuje s branami posílajícími data do cloudu,
odkud se zašlou kam zákazník žádá. Společnost Ingenu je jediným vývojářem 
dané technologie a zároveň jediným producentem příslušného hardwaru.
% Tak stanice, nebo brány?

Použitá technologie se nazývá RPMA --- Random Phase Multiple Access. Šířka 
využitého pásma je 80MHz, maximální komunikační rychlost je 80kbps. Spolehlivost
komunikace je pak zajištěna potvrzováním každé odeslané zprávy. Výrazným 
rozdílem oproti jiným technologiím je uváděný počet zařízení, které dokáže jedna 
stanice obsloužit --- dle výrobce až 2 miliony, tj. ekvivalent 18 LoRa bran či 
70 Sigfox bran. Standard podporuje dynamické přizpůsobení vysílacího výkonu a 
komunikační rychlosti v závislosti na aktuální vzdálenosti, míře rušení atd., 
díky čemuž výrobce uvádí výdrž zařízení až desítky let bez výměny baterií.

Ve srovnání s technologiemi používajícími sub-GHz pásma obnáší použití frekvencí 
v pásmu 2.4GHz nižší dosah při stejném vysílacím výkonu, vyšší ztráty v 
překážkách, vyšší spotřebu energie a podstatně vyšší zarušení --- ve stejném 
pásmu pracuje WiFi, Bluetooth aj.

%\href{https://www.ingenu.com/technology/}{https://www.ingenu.com/technology/},
%kniha

\section{Weightless}

\textbf{Existuje to stále?}

Weightless je otevřený standard pro LPWAN komunikaci. Je však patentován, a pro 
jeho provoz je třeba licence a kvalifikace komunikujícího zařízení. Tu provádí 
organizace Weightless SIG. Od prvního uvedení vznikly tři verze standardu --- 
W, N a P.

Původní verze Weightless W používá více různých modulací (např. DBPSK, stejně 
jako Sigfox) a pracuje v tzv. white-space --- v mezerách mezi přidělenými 
frekvenčními pásmy používanými komerčním televizním vysíláním. Topologie je opět 
podobná 
mobilním sítím, přenosové rychlosti se pohybují od 1kbps do 1Mbps. Z důvodu 
nedostupnosti užívaných frekvencí či nemožnosti jejich využití v mnoha oblastech 
byl vyvinut standard N.

Weightless N komunikuje v bezlicenčních pásmech, komunikace probíhá pouze 
jednosměrně od zařízení ke stanici. Rychlost komunikace je do 100kbps, dosah se 
pohybuje kolem 3km. Pro umožnění komunikace v obou směrech byl vyvinut třetí 
standard Weightless P, užívající GMSK a QPSK modulace, opět v bezlicenčních 
pásmech. Komunikace je tedy již obousměrná, umožňující (avšak nevyžadující) 
potvrzování odeslaných zpráv. Ve srovnání s Weightless N je dosah je nižší, 
zhruba kolem 2km, komunikační rychlost je pak podobná --- kolem 100kbps.

%{[}z knihy a wiki, stránky nedostupné.{]}

\section{Otevřené standardy v pásmu LTE}

LPWAN standardy NB-IoT a LTE-M jsou provozovány v pásmech LTE. Díky tomu je pro 
provozovatele telekomunikačních sítí snadné jejich nasazení, spočívající pouze 
v programové úpravě stávajících stanic. Oba standardy jsou otevřené, poskytované
skupinou 3GPP. Dosah je velmi závislý na pokrytí oblasti signálem, při přímé
viditelnosti může dosahovat až desítek kilometrů.

\subsection{NB-IoT}

NB-IoT je ve srovnání s LTE-M vhodné k provozu zařízení posílajících malé 
množství dat v delších časových intervalech. Používá modulaci OFDMA pro 
downlink, a SCFD uplink. Obousměrný provoz je podporován v režimu half-duplex. 
Jedna stanice je schopna obsluhovat až 50000 zařízení. Zařízení posílají zprávy 
o délce až 1600B, rychlostí max. 127kbps od stanice k zařízení a 159kbps od 
zařízení ke stanici. Z hlediska spotřeby energie je nevýhodou oproti jiným
technologiím nutnost udržovat spojení se sítí i v případě kdy zařízení 
komunikovat nepotřebuje.
% definovat downlink a uplink

\subsection{LTE-M}

LTE-M je v porovnání s NB-IoT vhodné k použití v kritických aplikacích, kdy je 
nutná komunikace v reálném čase. Ze všech LPWAN technologií má vůbec nejvyšší 
datovou propustnost. Jedna stanice dokáže obsluhovat přes 100000 zařízení. 
Přenosové rychlosti mohou špičkově dosahovat až 4Mbps dowlink, 7Mbps uplink. 
Obousměrný provoz je možno realizovat polovičním či plným duplexem. Standard 
podporuje i přenášení hlasu, avšak pouz v případech standardního pokrytí
(v kontrastu s rozšířeným pokrytím).

%https://en.wikipedia.org/wiki/Narrowband_IoT, 
%https://en.wikipedia.org/wiki/Narrowband_IoT,
%https://www.sierrawireless.com/iot-blog/lte-m-vs-nb-iot/

\section{DASH7}

DASH7 je zcela otevřený standard spravovaný organizací DASH7 Alliance, založený 
na standardu ISO/IEC 18000-7, popisujícím 433MHz RFID. DASH7 je tedy provozován 
v bezlicenčním sub-GHz pásmu. Topologie se skládá ze stanic, koncových zařízení 
a volitelně také subkontrolérů, které rozšiřují dosah stanic tím že na ně 
přeposílají přijaté zprávy. Mimo to je možná vzájemná komunikace mezi koncovými
zařízeními. Standard definuje všechny vrstvy ISO/OSI modelu, nikoli jen fyzickou 
a linkovou vrstvu. Komunikace je šifrovaná pomocí AES-128 algoritmu, nejvyšší 
rychlost je 167kbps. Podporované topologie jsou uzel-uzel, hvězdice, strom.